%%%%%%%%%%%%%%%%%%%%%%%%
%
%   Thesis template by Youssif Al-Nashif
%
%   May 2020
%
%%%%%%%%%%%%%%%%%%%%%%%%

\section{Graph Kernels for Similarity}

\hspace*{0.3cm} Graph kernels can be used to compare the similarity of two graphs. The development of these methods arose out of the need to determine if graphs were isomorphic in a faster way. The solution was a graph kernel which produces a scalar value for how similar two, or more, graphs are. The result of a graph kernel is a matrix where the similarity of graphs $i$ and $j$ is in the kernel's $i$-th row and $j$-th column entry. This resulting matrix can be used in a variety of ways, but it will be used as a distance matrix in applications here. \\

Amongst graph kernels which consider edge or vertex labels in their computation, various surveys and studies have found edge label histogram kernels or vertex label histograms to be the most efficient. They may not out perform the other methods, like a Weisfeiler-Lehman or other subgraph methods, but they are computationally cheap. The datasets being studied here are both large: one contains many smaller graphs, and one contains 48 very large graphs. So for this study, computation efficiency was prioritized. \\

The edge label histogram kernel is the graph kernel that was chosen for these datasets, and it can be computed using either a linear kernel or a Guassian radial basis function (RBF) kernel between the edge label histograms. \\
An edge label histogram is defined as $\vec{g} = (g_1,g_2, ... g_i)$ such that $g_i = | \{ (u,v) \in E | \phi(u,v) = i \} |$ for each $i$ \cite{sugiyama2015halting}. Where $g_i$ is a histogram bin for a unique edge label's magnitude, $E$ is the set of edges, and $\phi$ is a function that maps each label to a scalar value in the range of unique values. The edge label histograms are then passed through a kernel, either a linear kernel or a Gaussian RBF kernel.\\
Computation using a linear kernel takes two graphs, $G$ and $G'$, and uses their edge label histograms $\vec{g}$ and $\vec{g'}$. The kernel is computed as:

\begin{equation}
K(\vec{g},\vec{g'}) = \vec{g}^{T}\vec{g'}
\end{equation}

The resulting value is stored in the graph kernel matrix as the measure of similarity between the two graphs in the corresponding row and column for the pair \cite{sugiyama2015halting}.

Alternatively, the Gaussian RBF kernel takes the edge label histograms of $G$ and $G'$, that we call $\vec{g}$ and $\vec{g'}$, and the kernel is computed as:


\begin{equation}
K(\vec{g},\vec{g'}) = e^{- \left( \frac{||\vec{g}- \vec{g'}||^2}{2 \sigma^2} \right) }
\end{equation}

The resulting value is stored in the graph kernel matrix as the measure of similarity between the two graphs in the corresponding row and column for the pair. Through either of these kernels, we obtain a kernel of dimensions $n \times n$ for a list of $n$ graphs. This kernel can then be used for clustering methods.

% https://papers.nips.cc/paper/2015/file/31b3b31a1c2f8a370206f111127c0dbd-Paper.pdf


