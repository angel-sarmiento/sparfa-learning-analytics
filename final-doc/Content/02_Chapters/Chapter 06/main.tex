%%%%%%%%%%%%%%%%%%%%%%%%
%
%   Thesis template by Youssif Al-Nashif
%
%   May 2020
%
%%%%%%%%%%%%%%%%%%%%%%%%

\chapter{Future Work}

As stated above,  the entire investigation is built on open-source technologies in the R ecosystem with all code published 
to Github. This allows the entire project to be cloned and iterated upon through future explorations of the techniques 
discussed. One specific upside of this is that the code can easily applied to course data from other degree programs at FPU. A simple line of code to a function allows the work here to be applied to any department, with data cleaning and utility functions that can be iterated on. See the appendix below for more.

Some future implementations of this work can include approaches like BERTScores and OK scores as alternative distance 
approaches. Sparse Factor Analysis for an alternative to the network association graph creation. And other non-linear forms of dimensionality reduction.  There is also the potential for this data to be available to students, either through a web application or publicly available GUI for use in course planning for their academic success. 
It would also be feasible to gain data on question-concept association graphs \cite{willcox_network_2017} through sufficiently large databases of questions 
given to students over time that are characterized by the topics covered by those questions. This could potentially link student 
performance to their overall response to plans of study to adjust even further. 
