%%%%%%%%%%%%%%%%%%%%%%%%
%
%   Thesis template by Youssif Al-Nashif
%
%   May 2020
%
%%%%%%%%%%%%%%%%%%%%%%%%

\addcontentsline{toc}{chapter}{Abstract}
\begin{abstract}

In this work, we aimed to generate course-concept associations that can be pivotal in future course and curriculum design. To this end, a series of investigations involving 
Text Mining and Unsupervised learning techniques were developed and evaluated.  Learning Analytics is a field that involves the measurement, collection, analysis, and reporting of data
relevant to optimizing learning and learning environments. Machine learning aids in this, being widely known to be effective when applied to learning analytics in online 
and offline class structures. 
Through the use machine learning based analytics methods such as Multi-Dimensional Scaling (MDS) and Latent Dirichlet Allocation this work aims to provide two approaches: one approach 
for segmenting course catalogs and another for validating those segments. This application will be applied to a primarily in-person University. This allows these 
methods to be applied to potentially better advise students in University coursework for their long term goals by alleviating the pressure of learning a vast set of concepts 
from courses in which they might only need a small fraction of. The focus on this project was the course catalog at Florida Polytechnic University, where Document-term matrices, 
course-concept association graphs, and dimensionality reduction methods were constructed to investigate the future of catalogue design,  by unifying courses by their text 
information.  The work generated reusable templates with open-source technologies for applying these methodologies across the entire catalog with dynamic package management 
for reproducibility. This can be used to further analyze the planning of course work and catalogs, while pointing students toward important associations between their current 
coursework and weaknesses they may have in other concepts. The insights gained from this research project will help drive future developments and iterations on the application to aid 
in the University's rapid development. 

\end{abstract}